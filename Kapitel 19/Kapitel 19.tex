\subsection{Arbejde udført under volumeændring}
	\begin{align}
		W=\int_{V_1}^{V_2}p\,dV
	\end{align}
	\arbejde\\
	\volumen\\
	\tryk

	\subsubsection{Konstant tryk}
		\begin{align}
			W=p(V_2-V_1)
		\end{align}
		\arbejde\\
		\tryk\\
		\volumen

\subsection{Intern energi og termodynamikkens først lov}
	\begin{align}
		\Delta U=Q-W
	\end{align}
	\internenergiendr\\
	\varme\\
	\arbejde

	\subsubsection{Infinitesimal ændring i stadie}
		\begin{align}
			dU=dQ-dW
		\end{align}
		$U$: Interne energi i termodynamisk system ($J$)\\
		\varme\\
		\arbejde
	
\subsection{Typer af termodynamiske processer}
	\begin{tabular}{|c|ccc|}
		\hline
		Process&$Q$&$W$&$\Delta U$\\
		\hline
		Adiabatisk&0&&$-W$\\
		Isokorisk&&0&$Q$\\
		Isobarisk&&$p(V_2-V_1)$&\\
		Isotermisk&&$\int_{V_1}^{V_2}p\,dV$&\\
		\hline
	\end{tabular}

\subsection{Varmekapacitet af en ideal gas}
	\begin{align}
		C_p=C_V+R
	\end{align}
	$C_p$: Molare varmekapacitet ved konstant tryk ($\frac{J}{mol K}$)\\
	$C_V$: Molare varmekapacitet ved konstant volumen ($\frac{J}{mol K}$)\\
	\idealgaskonst

	\begin{align}
		\gamma=\frac{C_p}{C_V}
	\end{align}
	$\gamma$: Ratio af varmekapacitet\\
	$C_p$: Molare varmekapacitet ved konstant tryk ($\frac{J}{mol K}$)\\
	$C_V$: Molare varmekapacitet ved konstant volumen ($\frac{J}{mol K}$)

\subsection{Adiabatisk process for en ideal gas}
	\begin{align}
		W=nC_V(T_1-T_2)
	\end{align}
	\arbejde\\
	\mol\\
	$C_V$: Molare varmekapacitet ved konstant volumen ($\frac{J}{mol K}$)\\
	\tempk

	\begin{align}
		W=\frac{C_V}{R}(p_1V_1-p_2V_2)=\frac{1}{\gamma-1}(p_1V_1-p_2V_2)
	\end{align}
	\arbejde\\
	$C_V$: Molare varmekapacitet ved konstant volumen ($\frac{J}{mol K}$)\\
	\idealgaskonst\\
	\tryk\\
	\volumen\\
	$\gamma$: Ratio af varmekapacitet