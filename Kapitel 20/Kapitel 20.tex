\subsection{Varmemaskiner}
	\subsubsection{Termisk effektivitet}
		\begin{align}
			e=\frac{W}{Q_H}
		\end{align}
		\effektivitet\\
		\arbejde\\
		\varmeV

		\begin{align}
			e=1+\frac{Q_C}{Q_H}
		\end{align}
		\effektivitet\\
		\varmeK\\
		\varmeV

\subsection{Intern forbrændingsmotor}
	\subsubsection{Ottocyklussen}
		\begin{align}
			e=1-\frac{1}{r^{\gamma-1}}
		\end{align}
		\effektivitet\\
		$r$: Kompremeringsratio\\
		$\gamma$: Ratio af varmekapacitet

\subsection{Kølskabe}
	\begin{align}
		K=\frac{\vert Q_C\vert}{\vert W\vert}=\frac{\vert Q_C\vert}{\vert Q_H\vert-\vert Q_C\vert}
	\end{align}
	\kolydkoef\\
	\varmeK\\
	\arbejde\\
	\varmeV

	\begin{align}
		K=\frac{H}{P}
	\end{align}
	\kolydkoef\\
	\varmehast\\
	$P$: Power ($W$)

\subsection{Termodynamikkens anden lov}
	Det er umuligt at lave en proecs, som absorbere varme fra et reservoir ved en temperatur og lave varmen helt om til mekanisk arbejde, med det samme slutstaie som den begyndte i.

	\subsubsection{Termodynamikkens anden lov som kølemaskine}
		Det er umuligt at rykke varme fra et koldere legeme til et varmere legeme.
	
\subsection{Carnot cyklus}
	En cyklus der udelukkende består af isoterme og adiabatiske processor (den mest effektive cyklus)

	\subsubsection{Carnot effektivitet}
		\begin{align}
			e_{Carnot}=1-\frac{T_C}{T_H}=\frac{T_H-T_C}{T_H}
		\end{align}
		\carnoteffekt\\
		\tempV\\
		\tempK
	
	\subsubsection{Carnot kølemaskine}
		\begin{align}
			K=\frac{T_C}{T_H-T_C}
		\end{align}
		\kolydkoef\\
		\tempK\\
		\tempV

\subsection{Entropi}
	\subsubsection{Entropi i en revesibel proces}
		\begin{align}
			\Delta S=\int_1^2\frac{dQ}{T}
		\end{align}
		\entropi\\
		\varme\\
		\tempk
	
	\subsubsection{Entropi i en cyklisk reversibel proces}
		\begin{align}
			\Delta S=\int\frac{dQ}{T}=0
		\end{align}
		\entropi\\
		\varme\\
		\tempk
	
	\subsubsection{Termodynamikkens anden lov som entropi}
		Ingen proces kan lade sig gøre, hvor den totale entropi falder, når alle systemer der tager del i processen er inkluderet
	
\subsection{Entropi i mikroskala}
	\begin{align}
		S=k\ln w
	\end{align}
	$S$: Entropien i mikroskopiske termer ($\frac{J}{K}$)\\
	\boltzkonst\\
	$w$: Antallet at mikroskopiske stadier for et makroskopisk stadie