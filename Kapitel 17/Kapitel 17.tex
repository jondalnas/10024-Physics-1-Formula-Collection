\subsection{Temperatur og termisk ligevægt}
	\textbf{Termodynamikkens nulde lov}: Hvis C er i termisk ligevægt med A og B, så må A og B være i ligevægt med hinanden.\\
	\textbf{Kriterier for termiskligevægt}: To systemmer er i termiskligevægt hvis og kun hvis, begge deres temperaturer er det samme.\\

\subsection{Termisk udvidelse}
	\subsubsection{Linear udvidelse}
		\begin{align}
			\Delta L=\alpha L_0\Delta T
		\end{align}
		$\Delta L$: Længdeændring ($m$)\\
		\coeflinexp\\
		$L_0$: Begyndelseslængde ($m$)\\
		\tempk

	\subsection{Voluemn udvidelse}
		\begin{align}
			\Delta V=\beta V_0\Delta T
		\end{align}
		\volumen\\
		\coefvolexp\\
		\tempk
	
	\subsection{Termisk stress}
		\begin{align}
			F=-Y\alpha\Delta A
		\end{align}
		$F$: Termiske stress, når længde af materiale fastholdes ($N$)\\
		\youngmod\\
		\coeflinexp\\
		\tempk\\
		\areal
	
\subsection{Varmes størrelse}
	\subsubsection{Specifik varme kapacitet}
		\begin{align}
			Q=mc\Delta T
		\end{align}
		\varme\\
		\masse\\
		\secvarmcap\\
		\tempk
	
	\subsubsection{Molar varme kapacitet}
		\begin{align}
			Q=nC\Delta T
		\end{align}
		\varme\\
		\mol\\
		$C$: Molar varme kapacitet ($\frac{J}{mol K}$)\\
		\tempk

\subsection{Kalometri}
	\subsubsection{Latent varme (faseskifte)}
		\begin{align}
			Q=mL
		\end{align}
		\varme\\
		\masse\\
		\latent

\subsection{Varmeoverførsel}
	\begin{align}
		H=\frac{dQ}{dt}=kA\frac{T_H-T_C}{L}
	\end{align}
	\varmehast\\
	\varme\\
	\termkond\\
	\areal\\
	\tempV\\
	\tempK\\
	\lengde

	\begin{align}
		H=A\frac{T_H-T_C}{R}
	\end{align}
	\varmehast\\
	\areal\\
	\tempV\\
	\tempK\\
	\termmod

	\begin{align}
		R=\frac{L}{k}
	\end{align}
	\termmod\\
	\lengde\\
	\termkond

	\subsubsection{Stråling}
		\begin{align}
			H=Ae\sigma T^4
		\end{align}
		\varmehast\\
		\areal\\
		\emis\\
		\stefboltzkonst\\
		\tempk

	\subsubsection{Absorbans og stråling}
		\begin{align}
			H=Ae\sigma(T^4-T_s^4)
		\end{align}
		\varmehast\\
		\areal\\
		\emis\\
		\stefboltzkonst\\
		$T$: Absolutte overfladetemperatur ($K$)\\
		$T_s$: Absolutte omgivelsestemperatur ($K$)