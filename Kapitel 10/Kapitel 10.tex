\subsection{Kraftmoment}
	\begin{align}
		\tau=Fl
	\end{align}
	\kraftmom\\
	$F$: Kraften parallelt på distancen $l$ ($N$)\\
	$l$: Kraftpåvirkningens distance fra omdrejningspunkt ($m$)

	\begin{align}
		\boldsymbol{\tau}=\mathbf{r}\times\mathbf{F}
	\end{align}
	\Kraftmom\\
	\Kraft\\
	$\mathbf{r}$: Vektor fra omdrejningspunkt til kraftpåvirkningen ($m$)

\subsection{Kraftmoment og acceleration}
	\begin{align}
		\sum\tau=I\alpha
	\end{align}
	\Kraftmom\\
	\inertimom\\
	\vinkelacceleration

\subsection{Kombineret rotation og translatorisk energi}
	\begin{align}
		K=\frac{1}{2}Mv^2+\frac{1}{2}I\omega^2
	\end{align}
	\kinenergi\\
	\masse\\
	\vel\\
	\inertimom\\
	\vinkelhast

\subsection{Arbejde og effekt i rotation}
	\begin{align}
		W=\int_{\theta_1}^{\theta_2}\tau\,d\theta
	\end{align}
	\arbejde\\
	\vinkel\\
	\kraftmom

	\begin{align}
		P=\tau\omega
	\end{align}
	\effekt\\
	\kraftmom\\
	\vinkelhast

\subsection{Impulsmoment}
	\begin{align}
		\mathbf{L}=\mathbf{r}\times\mathbf{p}=\mathbf{r}\times m\mathbf{v}
	\end{align}
	\Impulsmom\\
	$\mathbf{r}$: Vektor fra omdrejningspunkt til impulset ($m$)\\
	\Impuls\\
	\masse\\
	\Vel

	\begin{align}
		\mathbf{L}=I\boldsymbol{\omega}
	\end{align}
	\Impulsmom\\
	\inertimom\\
	\vinkelhast

	\begin{align}
		\sum\boldsymbol{\tau}=\frac{d\mathbf{L}}{dt}
	\end{align}
	\Kraftmom\\
	\Impulsmom\\
	\tid

	\subsubsection{Konservering af impulsmoment}
		Hvis summen af alle kraftmomenter er 0, så er impulsmomentet i systemet konstant
		\begin{align}
			\sum\frac{d\mathbf{L}}{dt}=0
		\end{align}
		\Impulsmom

