\subsection{Moment og impuls}
	\begin{align}
		\mathbf{p}=m\mathbf{v}
	\end{align}
	\Moment\\
	\masse\\
	\Vel

	\subsubsection{Newtons anden lov i momenter}
		\begin{align}
			\sum\mathbf{F}=\frac{d\mathbf{p}}{dt}
		\end{align}
		\Kraft\\
		\Moment\\
		\tid
	
	\subsubsection{Impulser}
		\begin{align}
			\mathbf{J}=\sum\mathbf{F}\Delta t
		\end{align}
		\Impuls\\
		\Kraft\\
		\tid

		\begin{align}
			\mathbf{J}=\int_{t_1}^{t_2}\sum\mathbf{F}\,dt
		\end{align}
		\Impuls\\
		\Kraft\\
		\tid

	\subsubsection{Konservering af moment}
		Hvis summen af alle krafter er 0, så er momentet i systemet konstant (også kaldt elastisk kollision)
		\begin{align}
			\sum\mathbf{p}_f=\sum\mathbf{p}_e
		\end{align}
		$\mathbf{p}_f$: Impulserne før kollision ($\frac{kg m}{s}$)\\
		$\mathbf{p}_e$: Impulserne efter kollision ($\frac{kg m}{s}$)
	
	\subsubsection{Massemidtpunkt}
		\begin{align}
			\mathbf{r}_{cm}=\frac{\sum m\mathbf{r}}{\sum m}
		\end{align}
		$\mathbf{r}_{cm}$: Massemidtpunkt\\
		$\mathbf{r}$: Retningsvektor til en masse\\
		$m$: Masse ved et givent punkt i et legeme ($kg$)